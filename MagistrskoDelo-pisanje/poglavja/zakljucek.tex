\section{Zaključek}

Idejo vodenja in sodelovanja bi lahko še nekoliko razvili, a se zdi primerna za preizkušanje v resničnem svetu. V primeru, da strošek robotov ne bi bil prevelik, se zdi uporaba treh ovčarjev najboljša, saj je dovolj zanesljiva, četrti in peti ovčar pa pri naših nastavitvah simulacij nista posebno izboljšala rezultata.

Naš model sodelovanja dovoljuje uporabo enega vsevednega robota, ki ostalim robotom pošilja informacije o postavitvi ovc. S tem lahko imamo namesto skupine dragih robotov le enega dragega in nekaj cenejših, pri čemer dražji robot cenejše upravlja. Ta koncept, ki je v ozadju interakcij med psom in ovcami, lahko močno zniža cene pri več različnih problemih v robotiki.

Možnosti nadaljnjega dela so široke. Poleg izboljšave modela z adaptivnim genom bi število ovc poslali proti neskončnosti in čredo obravnavali kot sistem delcev. Tako bi se ukvarjali s porazdelitvami namesto z dejanskimi lokacijami agentov, kar lahko razumemo kot razlito tekočino na vodi. Poleg tega bi na pašnik lahko dodali ovire in vodo, kar bi v simulacije vneslo dodatne omejitve in dinamike. Poleg tega bi lahko naredili naključne oblike pašnikov, dodali volkove (grožnjo ovcam), pred katerimi bi morali ovčarji varovati čredo. S posplošitvijo v 3D prostor pa bi lahko simulirali tudi vodenje jate ptic in ideje preizkusili še v drugih domenah.
