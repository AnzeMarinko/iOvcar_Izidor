\section{Zaključek}

Idejo vodenja in sodelovanja bi lahko še nekoliko razvili, a zdi se primerna za preizkušanje v resničnem svetu. V primeru, da strošek robotov ne bi bil prevelik, se zdi uporaba treh ovčarjev najboljša, saj je dovolj zanesljiva, četrti in peti ovčar pa pri naših pogojih nista posebno izboljšala rezultata.

Možnosti nadaljnjega dela so široke. Poleg izboljšave modela z adaptivnim genom, bi število ovc poslali proti neskončnosti in čredo obravnavali kot sistem delcev. S tem bi se ukvarjali s porazdelitvami namesto dejanskimi lokacijami agentov, kar lahko razumemo kot razlito tekočino na vodi. Poleg tega bi na pašnik lahko dodali ovire in vodo, kjer ovčarji ali pa ovce ne želijo hoditi. Poleg tega bi lahko naredili naključne oblike pašnikov, dodali volkove (negativne grožnje ovcam), pred katerimi bi morali ovčarji varovati čredo. S posplošitvijo v 3D prostor pa bi lahko simulirali tudi vodenje jate ptic in ideje preizkusili še v drugih domenah.
