\section{Uvod}

Živali se pogosto zbirajo in gibljejo v bolj ali manj strnjenih skupinah. Skupini sesalcev iste vrste, posebej iz reda kopitarjev, rečemo čreda, skupina ptic ali rib je jata. Sicer pa se živali zbirajo tudi v tako imenovane trope, roje ipd. Z zbiranjem v skupino je posamezna žival bolj varna pred plenilci in se tudi lažje pari za ceno večje opaznosti skupine v primerjavi s posamezno živaljo, hitrejšim prenosom bolezni znotraj skupine ter za ceno manjše količine hrane na posamezno žival v njeni okolici.

Človek je za varovanje, zbiranje in vodenje črede ovc, da se je ta varno in hitreje premikala po pašniku ter tako varno popasla večje območje, udomačil pse ovčarje. Ovčarjevo obnašanje je pravzaprav prevzgojeno plenilsko obnašanje. V Sloveniji je edina avtohtona pasma kraški ovčar ali kraševec, ki ga je omenil že Janez Vajkard Valvasor v knjigi Slava Vojvodine Kranjske. Je pa kraški ovčar po značaju bolj pastirski pes, kar pomeni, da je umirjen, a ves čas pozoren na nevarnosti, saj bolj kot zbira čredo varuje~\cite{krasevec}.
Ovčarji so sami sposobni zbrati in voditi veliko čredo, lahko celo več sto ovc, saj jim pri tem ključno pomaga močan čredni nagon pri ovcah in strah ovc pred njimi. Podobno situacijo, kjer se skupina živali, ljudi ali drugih delcev (z eno besedo agentov) giblje v podobni smeri, opazimo tudi drugje v naravi in včasih bi nam koristilo, da bi imeli udomačeno žival tudi za druge podobne probleme. 

Tak primer so recimo jate ptic v okolici letališč. Jata ptic namreč lahko ob trku poškoduje letalo, kar se večinoma zgodi ob vzletanju ali pristajanju, saj večina ptic leti nizko, kar lahko vodi celo do strmoglavljenja. V ZDA naj bi bilo med letoma 1990 in 2010 že samo prijavljenih trkov s pticami kar 121 000, na svetu pa naj bi bilo med letoma 1988 in 2012 najmanj 200 smrti zaradi tovrstnih nesreč~\cite{letalske-nesrece}. Leta 2019 je v Etiopiji zaradi ptic v eni sami letalski nesreči umrlo 157 ljudi~\cite{letalske-nesrece2}. Najbolj znana nesreča s srečnim izidom pa je bila "Čudež na Hudsonu" v New Yorku leta 2009, ko je v nesreči dobro minuto po vzletu prišlo do trka z jato ptic. Preživelo je vseh 155 potnikov~\cite{letalske-hudson}. Pilot je kljub odpovedi obeh motorjev pristal in to kar na reki Hudson. O tem dogodku je posnet tudi film Sully~\cite{letalske-sully} imenovan po tem pilotu.

Za tovrstne probleme pa žal nimamo udomačene nobene vrste živali. Ideja je, da bi jih lahko nadomestili z uporabo robotov ali dronov. Zgledovali pa bi se po psih ovčarjih. V Novi Zelandiji so leta 2020 Boston Dinamics poskusili voditi ovce s pomočjo robotskega psa~\cite{robot-sheperd} in ugotovili, da se ovce podobno kot na psa odzivajo tudi na njihovega robota. Ta robot deluje brez človeškega sodelovanja, ampak ga za vodenje črede niso posebno izurili, mi pa si želimo najti uspešen algoritem, ki bo sposoben učinkovito pripeljati vse ovce v stajo v čim krajšem času. Uporaba pametnih ovčarjev robotov razvitih na podoben način bi lahko prav prišla tudi pri preusmerjanju jate ptic v okolici letališč, zbiranju razlite nafte na vodi ali pri vodenju panične množice ljudi na varno v kriznih situacijah.

Na letališčih že uporabljajo različne načine za preusmerjanje ptic. Da bi ptice vodili brez sodelovanja človeka, je korejska raziskovalna skupina že razvila algoritem za preusmerjanje jate ven iz območja letališča~\cite{bird-herd, letalske-chung, letalske-chung2}. Mi pa se bomo v tem delu osredotočili na vodenje črede ovc v stajo z uporabo umetne inteligence. Z uporabo umetne inteligence bi se ovčar robot lahko učil tudi tekom dela z vrsto živali, ki se obnaša kako drugače.

\subsection{Organizacija dela}

Različni avtorji so se že lotili modeliranja gibanja ovc in psa ovčarja kot agentov. V tem delu si bomo najprej ogledali dva modela gibanja ovc, ki sta ju predstavila avtorja Str{\"o}mbom~\cite{Stroembom} in Ginelli~\cite{Ginelli} s sodelavci. Prvi model bomo tudi nekoliko spremenili, da dobimo bolj naravno gibanje in tako skupno tri modele gibanja ovc. Obnašanje agentov je po vseh treh modelih dovolj različno, da bomo s tem lahko preizkusili robustnost algoritmov vodenja psa ovčarja.

Ogledali si bomo model gibanja psa ovčarja, ki so ga predlagali avtorji članka~\cite{Stroembom}. Modelu smo dodali tudi nekaj izboljšav in sodelovanje več psov z deljenjem črede glede na Voronoieve celice. Da bi uspeh algoritma še izboljšali v smislu hitrosti in uspešnosti vodenja ovc v stajo, smo parametre modela nastavili s pomočjo genetskega algoritma, ki temelji na ideji evolucije. Parametre smo nastavili v odvisnosti od velikosti črede, modela gibanja ovc in števila psov ovčarjev.

Ker pa se število ovc na pašniku sčasoma spreminja, ko ovce pridejo v stajo, želimo parametre prilagajati glede na trenutno in ne le glede na začetno stanje. Z uporabo spodbujevanega učenja (umetna inteligenca), natančneje PPO algoritma z nevronsko mrežo, si želimo dobljeni algoritem še izboljšati. Učenje smo pospešili z demonstracijami.

Na koncu pa si bomo ogledali še program \textit{iOvčar IZIDOR} za izvajanje simulacij in učenja, ki smo ga naredili v programu Unity in za umetno inteligenco uporablja paket ML Agents. Program je napisan v programskem jeziku~\textit{C\#}. Predstavitvi programa sledi še pregled rezultatov in primerjava modelov vodenja.
